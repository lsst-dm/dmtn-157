\begin{table}
\tiny
\centering
\begin{tabular} {|c|c|c|c|c|c|}
\hline
Run & Setup & Max CPUs & Workflow walltime & Cumulative job time & Compute cost \\
\hline
20200731T011149+20200802T164424 & regular N1 workers     & 800  & 55 hrs & 308 days & \$1592 \\
20200804T005441+20200806T041934 & preemptible N1 workers & 1600 & 36 hrs & 275 days, 22 hrs & \$459  \\
20200806T215620+20200808T170615 & preemptible N2 workers & 1600 & 25 hrs & 213 days, 10 hrs & \$390  \\
20200811T172329                 & regular N2 workers     & 800  & 31 hrs & 300 days, 2 hrs  & \$1191 \\

20200814T002816+20200816T060623 & preemptible N2 workers & 1600 & 29 hrs & 208 days, 22 hrs & \$388 \\
\hline
\end{tabular}
\caption{
Run summary of the HSC-RC2 workflow with different setups.
The workflow wall time only includes Pegasus records, and may be dominated by random job failures and the rescue graph.
The costs are overestimates.
The Postgres instance size was increased between the 20200802T164424 run and the 20200804T005441 run; it stayed the same afterwards.
}
\label{tab:runSummary}
\end{table}
